\documentclass[a4paper,landscape,10pt]{article}
\usepackage[paper=a4paper,landscape,left=20mm,right=20mm,top=20mm,bottom=20mm]{geometry}
\usepackage{longtable}
\usepackage{fancyhdr}
\usepackage[pdftex]{color}
\usepackage{colortbl}
\definecolor{green}{rgb}{0.04,0.68,0.04}
\definecolor{orange}{rgb}{0.97,0.65,0.12}
\definecolor{red}{rgb}{0.75,0.04,0.04}
\definecolor{gray}{rgb}{0.86,0.86,0.86}

\usepackage[pdftex,
            colorlinks=true, linkcolor=red, urlcolor=green, citecolor=red,%
            raiselinks=true,%
            bookmarks=true,%
            bookmarksopenlevel=1,%
            bookmarksopen=true,%
            bookmarksnumbered=true,%
            hyperindex=true,% 
            plainpages=false,% correct hyperlinks
            pdfpagelabels=true%,% view TeX pagenumber in PDF reader
            %pdfborder={0 0 0.5}
            ]{hyperref}

\hypersetup{pdftitle={Coverage Report},
            pdfauthor={ReportGenerator - 5.1.19.0}
           }

\pagestyle{fancy}
\fancyhead[LE,LO]{\leftmark}
\fancyhead[R]{\thepage}
\fancyfoot[C]{ReportGenerator - 5.1.19.0}

\begin{document}

\setcounter{secnumdepth}{-1}
\section{Summary}
\begin{longtable}[l]{ll}
\textbf{Generated on:} & 04/12/2023 - 22:17:41\\
\textbf{Coverage date:} & 04/12/2023 - 22:17:41\\
\textbf{Parser:} & Cobertura\\
\textbf{Assemblies:} & 1\\
\textbf{Classes:} & 2\\
\textbf{Files:} & 2\\
\textbf{Covered lines:} & 3\\
\textbf{Uncovered lines:} & 6\\
\textbf{Coverable lines:} & 9\\
\textbf{Total lines:} & 28\\
\textbf{Line coverage:} & 33.3\% (3 of 9)\\
\textbf{Covered branches:} & 0\\
\textbf{Total branches:} & 0\\
\textbf{Covered methods:} & 1\\
\textbf{Total methods:} & 3\\
\textbf{Method coverage:} & 33.3\% (1 of 3)\\
\end{longtable}
\section{Risk Hotspots}
No risk hotspots found.
\section{Coverage}
\begin{longtable}[l]{|l|r|r|r|r|r|r|r|}
\hline
\textbf{Name} & \textbf{Covered} & \textbf{Uncovered} & \textbf{Coverable} & \textbf{Total} & \textbf{Line coverage} & \textbf{Branch coverage} & \textbf{Method coverage}\\
\hline
\textbf{ConsoleApp} & \textbf{3} & \textbf{6} & \textbf{9} & \textbf{28} & \textbf{33.3\%} & \textbf{} & \textbf{33.3\%}\\
\hline
ConsoleApp.Class1 & 3 & 3 & 6 & 21 & 50\% &  & 50\%\\
\hline
Program & 0 & 3 & 3 & 7 & 0\% &  & 0\%\\
\hline
\end{longtable}
\newpage
\section{ConsoleApp.Class1}
\subsection{Summary}
\begin{longtable}[l]{ll}
\textbf{Class:} & ConsoleApp.Class1\\
\textbf{Assembly:} & ConsoleApp\\
\textbf{File(s):} & \begin{minipage}[t]{12cm}{/home/runner/work/dotnet-app-test/dotnet-app-test/ConsoleApp/Class1.cs}\end{minipage} \\
\textbf{Covered lines:} & 3\\
\textbf{Uncovered lines:} & 3\\
\textbf{Coverable lines:} & 6\\
\textbf{Total lines:} & 21\\
\textbf{Line coverage:} & 50\% (3 of 6)\\
\textbf{Covered branches:} & 0\\
\textbf{Total branches:} & 0\\
\textbf{Covered methods:} & 1\\
\textbf{Total methods:} & 2\\
\textbf{Method coverage:} & 50\% (1 of 2)\\
\end{longtable}
\subsection{Metrics}
\begin{longtable}[l]{|l|r|r|r|}
\hline
\textbf{Method} & \textbf{Branch coverage} & \textbf{Cyclomatic complexity} & \textbf{Line coverage}\\
\hline
\textbf{Method1()} & 100\% & 1 & 100\%\\
\hline
\textbf{Method2()} & 100\% & 1 & 0\%\\
\hline
\end{longtable}
\subsection{File(s)}
\subsubsection{/home/runner/work/dotnet-app-test/dotnet-app-test/ConsoleApp/Class1.cs}
\begin{longtable}[l]{lrrll}
\textbf{} & \textbf{\#} & \textbf{Line} & \textbf{} & \textbf{Line coverage}\\
\cellcolor{gray} &  & \verb~1~ & & \verb~using System;~\\
\cellcolor{gray} &  & \verb~2~ & & \verb~using System.Collections.Generic;~\\
\cellcolor{gray} &  & \verb~3~ & & \verb~using System.Linq;~\\
\cellcolor{gray} &  & \verb~4~ & & \verb~using System.Text;~\\
\cellcolor{gray} &  & \verb~5~ & & \verb~using System.Threading.Tasks;~\\
\cellcolor{gray} &  & \verb~6~ & & \verb~~\\
\cellcolor{gray} &  & \verb~7~ & & \verb~namespace ConsoleApp~\\
\cellcolor{gray} &  & \verb~8~ & & \verb~{~\\
\cellcolor{gray} &  & \verb~9~ & & \verb~    public class Class1~\\
\cellcolor{gray} &  & \verb~10~ & & \verb~    {~\\
\cellcolor{gray} &  & \verb~11~ & & \verb~        public void Method1()~\\
\cellcolor{green} & 1 & \verb~12~ & & \verb~        {~\\
\cellcolor{green} & 1 & \verb~13~ & & \verb~            Console.WriteLine("Method1");~\\
\cellcolor{green} & 1 & \verb~14~ & & \verb~        }~\\
\cellcolor{gray} &  & \verb~15~ & & \verb~~\\
\cellcolor{gray} &  & \verb~16~ & & \verb~        public void Method2()~\\
\cellcolor{red} & 0 & \verb~17~ & & \verb~        {~\\
\cellcolor{red} & 0 & \verb~18~ & & \verb~            Console.WriteLine("Method2");~\\
\cellcolor{red} & 0 & \verb~19~ & & \verb~        }~\\
\cellcolor{gray} &  & \verb~20~ & & \verb~    }~\\
\cellcolor{gray} &  & \verb~21~ & & \verb~}~\\
\end{longtable}
\newpage
\section{Program}
\subsection{Summary}
\begin{longtable}[l]{ll}
\textbf{Class:} & Program\\
\textbf{Assembly:} & ConsoleApp\\
\textbf{File(s):} & \begin{minipage}[t]{12cm}{/home/runner/work/dotnet-app-test/dotnet-app-test/ConsoleApp/Program.cs}\end{minipage} \\
\textbf{Covered lines:} & 0\\
\textbf{Uncovered lines:} & 3\\
\textbf{Coverable lines:} & 3\\
\textbf{Total lines:} & 7\\
\textbf{Line coverage:} & 0\% (0 of 3)\\
\textbf{Covered branches:} & 0\\
\textbf{Total branches:} & 0\\
\textbf{Covered methods:} & 0\\
\textbf{Total methods:} & 1\\
\textbf{Method coverage:} & 0\% (0 of 1)\\
\end{longtable}
\subsection{Metrics}
\begin{longtable}[l]{|l|r|r|r|}
\hline
\textbf{Method} & \textbf{Branch coverage} & \textbf{Cyclomatic complexity} & \textbf{Line coverage}\\
\hline
\textbf{$<$Main$>$$(...)} & 100\% & 1 & 0\%\\
\hline
\end{longtable}
\subsection{File(s)}
\subsubsection{/home/runner/work/dotnet-app-test/dotnet-app-test/ConsoleApp/Program.cs}
\begin{longtable}[l]{lrrll}
\textbf{} & \textbf{\#} & \textbf{Line} & \textbf{} & \textbf{Line coverage}\\
\cellcolor{gray} &  & \verb~1~ & & \verb~// See https://aka.ms/new-console-template for more information~\\
\cellcolor{gray} &  & \verb~2~ & & \verb~using ConsoleApp;~\\
\cellcolor{gray} &  & \verb~3~ & & \verb~~\\
\cellcolor{red} & 0 & \verb~4~ & & \verb~Console.WriteLine("Hello, World!");~\\
\cellcolor{gray} &  & \verb~5~ & & \verb~~\\
\cellcolor{red} & 0 & \verb~6~ & & \verb~var class1 = new Class1();~\\
\cellcolor{red} & 0 & \verb~7~ & & \verb~class1.Method1();~\\
\end{longtable}
\end{document}